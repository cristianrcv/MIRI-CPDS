\section{Introduction}
\label{sec:Introduction}

The seminar is mainly about the Paxos algorithm which is used to gain consensus
in a distributed system. Paxos is a flexible and fault tolerant protocol for
solving the consensus problem, where participants in a distributed system
need to agree on a common value. For the algorithm correctness it doesn’t matter
what this value is, but it must ensure that a single one among the proposed values
is ever chosen.

The Paxos algorithm has three different processes: proposers, acceptors and learners
although all three are often included in one single process. This seminar pretends
to provide an implementation for the proposer and acceptor processes but excludes 
the learner process since it is not needed to reach a consensus. 

The main topic related to distributed systems that this seminar covers is about how to
gain consensus in a distributed system under the condition of perfect communication
and faulty processes.
