\section{Work Done}
\label{sec:WorkDone}

In this assignment we have implemented the PAXOS protocol 
in Erlang in order to see some basic functionality. The simulation scenario
consisted of a system with three \emph{Proposers} and five \emph{Acceptors}.
This dimensions are far away from real distributed systems, but in turn it allowed
us to understand easily the key steps of PAXOS. A graphical
user interface combined with a terminal provided all the necessary
information to check each step of the simulations.

First, we completed the provided source code (\emph{Paxy}) following the theory
explained in class. There were some implementation-related issues not covered
in theory that raised our interest. For example, it was interesting to see in the
code how \textit{Proposers} only consider messages from their current rounds, ignoring
the rest of them in their mailboxes even they were sent by them. This gives an idea
about the asynchronous nature of this problem. 

After completing the code, we added some delay to the acceptors replies in order to simulate latency in
the network, approaching this way more realistic conditions. In addition to this, we inserted message
dropping in order to simulate an unreliable transport channel. This will be explained
more in detail later.

This assignment also includes some improvements of the original version. We removed the sorry 
messages to understand how relevant are they inside the PAXOS protocol. We also
added fault tolerancy to our processes so they could store their information in a persistent storage 
device and recover it in case of a crash. We also separated the PAXY implementation to
run the acceptors and the proposers on different machines. Finally, we implemented an improvement based on the proper 
interpretation of sorry messages in order to finish PAXOS earlier. This will be discussed in detail
along the assignment.
